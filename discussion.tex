% !TeX root = sta_paper.tex

In this study, the main question was whether verbalization assists with other processes to influence visual memory performance. In that case, the application of articulatory suppression would be required to disengage that dimension so that a pure measure of visual memory performance can be obtained. Neither a straightforward descriptive analysis nor a state-trace analysis revealed evidence that participants engaged in verbalization as a strategy, even though the experimental design favored the use of verbalization even more than typical visual change detection tasks. The lack of a complex relationship between suppression, presentation type, and performance provides evidence that verbal recoding was not a strategy used by the participants in this task.

One caveat for the interpretation of these results is that state-trace analysis, like all methods, is limited by the resolution of the data. Detecting deviations from monotonicity in a curve depends on how finely points on the curve are measured. It is possible that with finer gradations of set size, we might be able to detect non-monotonicities that are not apparent in these data. However, visual inspection of the state-trace plots in Figure~\ref{fig:ST_plots} suggests that whatever effect of articulatory suppression exists is small; detecting such a small deviation from monotonicity would require finer gradations of sets size and more trials per set size. Although a small deviation from monotonicity may in fact exist, it is unlikely to have any substantial effect on measurements of visual working memory capacity.

Stimulus presentation duration is likely a crucial factor determining whether verbalization strategies are employed in visual memory tasks. Another possible reason for the lack of an effect of articulatory suppression might be that the presentation rate in the experiment was rather fast relative to the time it takes to verbalize information. However, the stimulus presentation timings we employed (100 ms) are typical of visual change detection tasks, which range from 8 ms per item \citep{Woodman:Vogel:2005} to as much as 500 ms per item \citep{Brockmole:etal:2008} in the visual change detection papers cited in our Introduction, and the actual time to articulate was closer to 300 ms in our study. For presentations as fast or faster than the 100 ms per item rate that we measured, it appears safe to assume that verbalization does not augment visual change detection performance. The abstractness of the stimuli employed also likely influences the extent to which verbalization occurs. Researchers employing nameable visual stimuli at a pace enabling verbalization should still consider employing precautionary articulatory suppression if their goal is to isolate specifically visual memory. However, based on our data, we conclude that in typical visual change detection, this precaution is unnecessary.

Finally, we emphasize that we do not dispute that verbal processes can in principle be applied to visual stimuli, and likely are under some circumstances \citep[e.g.][]{Brandimonte:etal:1992}. Our manipulations were designed specifically to uncover the extent to which verbalization was likely to influence visual memory performance under the conditions typical to visual change detection tasks. Our findings suggest that participants do not make use of this hypothetical second dimension (i.e., verbalization) in this task context. Consequently, enforcing precautionary articulatory suppression does not seem to be necessary to get interpretable data from visual change detection tasks.