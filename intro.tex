% !TeX root = sta_paper.tex

During his seminal experiments on human memory, Sperling noticed that many of his participants verbalized and repeated to-be-remembered material during rehearsal, even if the studied material was not aurally presented. \citet{Sperling:1967} pointed out that visual information can be verbalized and many people reported doing so. This reflection confirmed to cognitive psychologists interested in the structure and mechanisms underlying memory that presentation modality likely affects encoding. Like Sperling, they used this insight as a basis for inquiry into the nature of memory representations.

For example, \citet{Murray:1965} showed that saying visually-presented verbal stimuli out loud improves recall performance relative to mouthing them silently. However, this relationship only seems to persist if the visually-presented material can be verbalized effectively (e.g., verbal stimuli, nameable visual images). Attempts to verbalize stimuli that are difficult to describe succinctly and accurately (e.g., faces) might actually harm performance \citep{Schooler:Engstler-Schooler:1990}. \citet{Brandimonte:etal:1992} showed that verbal recoding can be detrimental to a subsequent mental rotation task when the remembered verbal label is not relevant or helpful. What such experiments suggest is that there is a strong tendency to verbally recode visually-presented information, and that in some cases verbal recoding can boost memory performance. This logic is consistent with multi-component models of working memory, which propose that separate short-term memory stores for phonological and visual information can be applied to a short-term memory task \citep{Baddeley:1986}. Naturally, if task-relevant information can be maintained simultaneously in two codes, one would expect memory performance to improve. 

However, the possibility of dual encoding is problematic if the goal is to measure capacity for visual information exclusively. \citet{Levy:1971} suggested a method of preventing such recoding via meaningless concurrent articulation. By repeating irrelvant syllables out loud during presentation and rehearsal of visual information, participants' ability to verbally recode visually-presented stimuli is restricted. This procedure is known as \emph{articulatory suppression} and is commonly used with visual change detection tasks specifically to prevent verbalization of visual stimuli \citep[e.g.][]{Allen:etal:2006, Brockmole:etal:2008, Delvenne:Bruyer:2004, Hollingworth:Rasmussen:2010, Logie:etal:2009, Makovski:Jiang:2008, Makovski:etal:2008, Matsukura:Hollingworth:2011, Treisman:Zhang:2006, vanLamsweerde:Beck:2012, Woodman:Vogel:2005, Woodman:Vogel:2008}. This precaution is undertaken explicitly to ensure that task performance reflects visual memory, rather than some combination of memory for visual images and verbal codes. 

While the use of articulatory suppression is common practice, there is also evidence that it does not have an effect on some visual change detection tasks \citep{Luria:etal:2010,Morey:Cowan:2004, Morey:Cowan:2005}. These studies imply that the precaution of employing articulatory suppression may be unnecessary: participants performed no better without articulatory suppression than with it, suggesting that verbal recoding is not the default strategy for visual change detection tasks as typically administered. \citet{Mate:etal:2012} note that articulation of task-relevant words can affect visual memory performance, but confirm the findings of \citet{Morey:Cowan:2004} that meaningless speech has no effect. Nonetheless, judging from the frequency with which researchers employ precautionary articulatory suppression during visual change detection, it appears to be uncritically accepted that prevention of verbalization of visual images is crucial to obtaining a pure measure of visual memory performance. 

Because meaningless articulation does not appear to affect visual change detection performance, we suspect that preventing verbal recoding is not always essential for measuring visual memory capacity. Because enforcing articulation adds a substantial burden to an experiment from both the participant's and the experimenter's point of view, it would be preferable to eliminate articulatory suppression from designs where it is not needed. 

We outline evidence from a large experiment -- thousands of trials per participant -- that suggests there is reason to believe that articulatory suppression has no discernible effect on performance in a visual change-detection task. The experiment was designed so that some change-detection conditions encourage verbalization; if participants are verbalizing the visual stimuli, one would expect that articulation would impair performance substantially more in these conditions. 